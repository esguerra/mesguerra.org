%===============================================================================
%  Preamble
%===============================================================================
\documentclass[10pt, oneside, pdftex]{article}
\usepackage[T1]{fontenc}
\usepackage[bitstream-charter]{mathdesign}
\usepackage[latin1]{inputenc}			% Input encoding
\usepackage{amsmath, amstext, amsfonts}		% Math
\usepackage{xcolor}
\definecolor{bl}{rgb}{0.0,0.2,0.6} 
\usepackage{sectsty}
\usepackage{url}
\usepackage{fancyvrb}
\usepackage{listings}
\usepackage{setspace}
\singlespacing
\usepackage[left=2.54cm,bottom=2.00cm,top=2.00cm,right=2.54cm]{geometry}
\usepackage[compact]{titlesec} 
\allsectionsfont{\color{bl}\scshape\selectfont}

%===============================================================================
%  Definitions
%===============================================================================
% Define a new command that prints the title only
\makeatletter							% Begin definition
\def\printtitle{%						% Define command: \printtitle
  {\color{bl} \centering \huge \sc \textbf{\@title}\par}}	% Typesetting
\makeatother							% End definition

\title{Pymol Tricks \\ 
	\large \vspace*{-10pt}A compilation of recipes\vspace*{10pt}}

% Define a new command that prints the author(s) only
\makeatletter							% Begin definition
\def\printauthor{%					% Define command: \printauthor
    {\centering \small \@author}}				% Typesetting
\makeatother							% End definition

\author{%
	Mauricio Esguerra \\
	mauricio.esguerra@gmail.com \\
	\vspace{20pt}
	}

% Custom headers and footers
\usepackage{fancyhdr}
	\pagestyle{fancy}					% Enabling the custom headers/footers
\usepackage{lastpage}	
	% Header (empty)
	\lhead{}
	\chead{}
	\rhead{}
	% Footer (you may change this to your own needs)
	\lfoot{\footnotesize \texttt{mesguerra.org} - Pymol Tricks}
	\cfoot{}
	\rfoot{\footnotesize page \thepage\ }%of \pageref{LastPage}}	% "Page 1 of 2"
	\renewcommand{\headrulewidth}{0.0pt}
	\renewcommand{\footrulewidth}{0.4pt}

% Change the abstract environment
\usepackage[runin]{abstract}			% runin option for a run-in title
\setlength\absleftindent{30pt}		% left margin
\setlength\absrightindent{30pt}		% right margin
\abslabeldelim{\quad}						% 
\setlength{\abstitleskip}{-10pt}
\renewcommand{\abstractname}{}
\renewcommand{\abstracttextfont}{\color{bl} \small \slshape}	% slanted text


%%% Start of the document
\begin{document}
%%% Top of the page: Author, Title and Abstact
\printtitle 

\printauthor

\begin{abstract}
\noindent Installing pymol nowadays is very easy in almost any Linux version with a
package management system like yum or apt-get. For more information on this the
pymolwiki at \url{http://pymolwiki.org} is the best resource. The following is 
just a compilation of tricks for pymol power-users.
\end{abstract}


\section*{ - General}
\definecolor{mygray}{rgb}{0.9,0.9,0.9}
%\lstset{language=Python,
%backgroundcolor=\color{mygray},
%frame=single,
%basicstyle=\footnotesize\sffamily}
\fvset{fontfamily=courier,
fontsize=\small,
frame=single}
%backgroundcolor=\color{mygray}}
%\begin{lstlisting}
\begin{Verbatim}
bg white                       # Turn background color to white
save rna6.pse                  # Save your working session to pse file
set valence, 1                 # Show double bonds.
h_add                          # Show hydrogens.  
pymol -qc bla.pml >& bla.log & # To invoke without X11 display
pymol -M                       # Invoke mono mode instead of stereo
                               # to avoid flicker in some screens.
unbond (id 8), (id 2)          # Remove a bond between atoms
bond (id 8), (id 2)            # Create a bond between atoms
select carbons, symbol c       # Select all carbon atoms and call them carbons.
color black, carbons           # Color the carbons selection black.
disable selectionname          # Hides/disables a loaded structure(selection).
enable selectionname           # Shows/enables a loaded structure(selection).
set internal_gui, 0            # Hide object control panel
\end{Verbatim}
%\end{lstlisting}

\section*{ - Labels}
\begin{Verbatim}
label (n;C1'), "%s" % (name)
set label_color, black
set label_font_id, 4
set label_size, 3
set label_position, [2,0,0]
show labels
\end{Verbatim}

\section*{ - Spheres}
\begin{Verbatim}
set sphere_scale, 0.25
\end{Verbatim}

\section*{ - Sticks}
\begin{Verbatim}
show sticks, selectionname
set stick_radius, 0.1
set stick_ball, on #Makes the stick command show ball and stick
set stick_ball_ratio, 1.5
\end{Verbatim}

\section*{ - Ribbons}
\begin{Verbatim}
set ribbon_color, marine
set ribbon_width, 2
set ribbon_sampling, 1
set ribbon_smooth, 1
\end{Verbatim}

\section*{ - Cartoon}
\begin{Verbatim}
cartoon putty
set cartoon_ladder_mode, 0
set cartoon_transparency, 0.5
set cartoon_ring_finder, 0
cartoon dumbbell
set cartoon_dumbbell_width, 0.2
set cartoon_dumbbell_radius, 0.4
set cartoon_color, marine, resi 34-67
set cartoon_color, blue, resi 68,78
show cartoon
\end{Verbatim}

\section*{ - Ray tracing}
\begin{Verbatim}
set ray_trace_mode, 3
set ray_trace_fog,0
set ray_shadows,0
set depth_cue, 0            #This one with care or it might look funky.
set antialias,1
set ray_trace_gain, 0.005
ray 1600,1200
png img1.png
\end{Verbatim}

\section*{ - Rotating}
"turn" to rotate all loaded objects, "rotate" for individual objects
\begin{Verbatim}
turn x, 10
\end{Verbatim}



Definition of Objects and Selections found at:

\url{http://pymolwiki.org/index.php/Objects_and_Selections}

\section*{ - Important Settings}

\fvset{frame=single, fontfamily=courier, fontsize=\small}
\begin{Verbatim}
orthoscopic (0 or 1) controls whether the OpenGL renderer uses the same
orthoscopic transformation as the renderer. You'll want to set this to 1
when preparing figures so that OpenGL and raytracing match pixel-for-pixel.

ambient (0.0-1.0) controls the ambient light intensity for both OpenGL and the
ray-tracer.

ambient_scale (float) controls the relative ambient intensity between OpenGL
and the ray-tracer.

antialias (0 or 1) generate a "smooth" image (best quality, but takes
4X as long).

direct (0.0-1.0) the planer light intesity originating from the camera.

gamma (0.1-2.0) gamma transformation applied after rendering is complete.

light (vector) the position of the light.

reflect (0.0-1.0) the planer light intesity originating from the light source.

spec_reflect (0.0-1.0) intensity of the specular reflection from the light.

spec_power (1-100) how localized is the specular reflection (higher=smaller).
\end{Verbatim}


\section*{Electrostatic Potential Maps}
Go to:
\url{http://nbcr-222.ucsd.edu/pdb2pqr_1.8/}

Upload your pdb or use one id you like.

Download the resulting pqr file

open in pymol


%\lstset{language=python, frame=single}
\fvset{frame=single, fontfamily=courier, fontsize=\small}
\begin{Verbatim}
color white
color blue, (pc; > 0.1)
color red,  (pc; < 0.1)
show surface
\end{Verbatim}

\section*{Ray Tracing}
To ray trace an image with an opaque background.
\begin{Verbatim}
# turn on transparent alpha channel
set ray_opaque_background, off
\end{Verbatim}

\section*{General Tips}
Shift-right\_click and slide controls fog intensity and after a point
it starts clipping.

\noindent Shift-wheel controls fog intensity and after a point it starts
clipping. 

\end{document}



