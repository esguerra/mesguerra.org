\part{3DNA}

\section{Program Output}

With 3-DNA  we can  get parameters evaluated  for the  single stranded
mode and  the double stranded mode.\\
The data obtained  from such are:\\

Single Stranded Mode:\\
\begin{itemize}
\item auxiliary.par\\
                - Local Base Reference Frames
\item bp\_helical.par\\  
                - Local Helical Parameters
\item bp\_step.par\\     
                - Local Step Parameters 
\item ref\_frames.dat

\item file.outs \\     
                - RMSD of bases\\
                - Overlap Area\\
		- local base step parameters\\
		- Chain and chi Torsion angles\\
		- Sugar conformational parameters\\
		- P--P \& C1\'{}--{}C1\'~ virtual bond distances\\
		- Helix Radius (P, O4\'~ \& C1\'~)\\
		- Position (Px, Py, Pz) and local helical axis vector (Hx, Hy, Hz)\\
\end{itemize}

Double Stranded Mode:\\

\begin{itemize}
\item auxiliary.par\\   
                - Same as above but per base-pair
\item bp\_helical.par\\  
                - Same as above but per base-pair
\item bp\_step.par\\     
                - Same as above but per base-pair
\item cf\_7methods.par\\ 
                - Parameters from other softwares
\item ref\_frames.dat\\
\item bp\_order.dat\\    
                - The base-pair information before reordering.
                 Contains the numbered info of which are the 
		 bases belonging to the helical regions.
\item col\_chains.scr\\  
                - Prepares chain colors.
\item col\_helices.scr\\ 
                - Assigns colors to helices.
\item bestpairs.pdb\\   
                - All the base-pairs that it finds given constraints.
\item hel\_regions.pdb\\ 
                - The regions that are considered as helical
                  So they must be close to A-DNA, B-DNA or
		  Z-DNA in one way or another, I would think.
\item hstacking.pdb\\   
                - It seems like it puts them in the same reference
                 frame so that they can be stacked
\item stacking.pdb\\    
                - Similar to hstacking. Don't know yet the difference
\item file.out\\        
                - All base-pair plus base-step parameters and more
\end{itemize}
