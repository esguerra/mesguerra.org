%Introduction to my work in Dr. Olson's Lab
\part{RNA Structure}
\section{Problem}

The work  in my rotation with  Professor Wilma Olson  started with the
formulation of a  question, which although it seems  ``easy'' at first,
it definitely is not,  so the  question is  if one  can in  some way
reduce the  experimental 3D structural information that  one has got  for the large
subunit  of  the ribosome,  in  a way  that  one  can interpret  ``per
region'' interactions in it.   So the  first thing is  to remember what RNA
structure is based on and see what  has been done and what is being done to solve this question.

\subsection{Base Interactions}
RNA Structure is based mainly  in two types of interactions, these are
stacking and hydrogen bonding between base pairs\cite{sponer1}, so for
the latter  one finds the  canonical Watson-crick (W-C) pairs  and there are also have  non  W-C  pairs, for  example  Hoogsteen and  reversed
Hoogsteen\cite{leontis1, saenger} as seen in the following figure.

\begin{figure}[htb]
\centering
\includegraphics[scale=0.15]{pairing.png}
\caption{Pairing Types}
%\label{Pairing Types Figure}
\end{figure}

So the  first step  that is done  in order  to be able  to rationalize
these kind of  interactions is to define a  reference frame, there are
various different standards described in the literature for these purpose  and these fact
gave rise  to conflicting  interpretations of nucleic  acid structures\cite{lu1}
being produced by a diverse  amount of software's\cite{lu2}, because of this it
has recently been proposed a standard reference frame\cite{olson1} to abolish these
conflict which has already been integrated in the NDB database.

\begin{figure}[htb]
\centering
\includegraphics[scale=0.4]{ideal.png}
\caption{Idealized standard base-pair}
%\label{Ideal B-P}
\end{figure}


Once  you  have a  standard  reference frame,  then  you  can find  an
orthogonal-axis to  the base-pair plane following the  right hand rule
whose direction goes from 5' to 3' following the sequence strand.  Now
one can start  taking into account stacking because  we have defined a
center for  the base-pair  which can  act as a  spinal cord.  With the
standard  reference  frame defined  it  is  possible  to calculate  3D
stucture parameters using the 3DNA program\cite{lu3}.

\subsection{More Interactions}
Of course  base interactions are not the  only ones in RNA,  it is also
important to  consider the  sugars and the  phosphate groups,  this is
done  by  3DNA allowing to  create structures which can form the input for M.M. or M.D. programs.







%I  started looking  at the  studies on  structural motifs  of  RNA, of
%course the  first paper that I looked  at was the one  of 3-DNA, then,
%after being  a little confused about which  are canonical watson-crick
%base  pair,  which  not  and  what  is  a  hoogstein  configuration  I
%remembered that P.  Hobza had  a great article where he introduced you
%to these,  so I started  searching again at  hobza's work and  I found
%some of his  papers describing the base pair  interaction and how they
%must be considered, so  this is what I have done for  today and I will
%be reading  these and  putting a  review every day  that I  come here.
%REMINDER: Have to look at what they gave to me!! over the weekend!!.






\begin{quote}
\textit{``Example on including Quotation''}
\end{quote}


\subsection{Subsection B}



\subsubsection{Sub Sub Section }


This is just to leave an example of a table which I might later find useful.


\begin{center}
%\begin{table}
\begin{tabular}{|l|c|c|}\hline
\multicolumn{3}{|c|}{TABLA 1: OBJETOS DEL ISM}\\ \hline\hline
Objeto                         & Densidad ($cm^{-3}$)  &
Temperatura(K)     \\ \hline Regiones $H^2$                 &
$\approx 200$         & $\approx 15$       \\ \hline Regiones HI
& $\approx 25$          & $\approx 120$      \\ \hline Regiones
HII                   & $\approx 15$          & $\approx 8000$
\\ \hline Nubes Interestelares Densas    & $10 - 1000$           &
$50-100$           \\ \hline Nubes Interestelares Difusas   &
$10^3 - 10^6$         & $< 50$             \\ \hline
\end{tabular}%\caption{Composici�n del ISM}
%\end{table}
\end{center}


\begin{center}
%\begin{table}
\begin{tabular}{|p{3,2cm}|c|c|p{3,2cm}|}\hline
\multicolumn{4}{|c|}{TABLA 2: COMPOSICI�N DE NUBES}\\ \hline\hline
{\sf \small Objeto} &{\sf \small Densidad($cm^{-3}$)} &{\sf \small
Temperatura(K)} &{\sf \small Principales Caracte-}
\\
& & &{\sf \small r�sticas Moleculares}
\\       {\bf Nubes Difusas}
& & &
\\\hline
{\sf \small Condiciones T�picas} & 100 & $50 - 100$ & Algunas
diat�micas
\\
{\sf \small Regiones Impactadas} & & $\leq 4000$ & $CH^+$
\\
{\sf \small Nubes Trasl�cidas} & 1000 & 10 -50 & Algunas
poliat�micas
\\
{\bf Nubes Densas} & & &
\\\hline
{\sf \small Material inter-centro} & 1000 & 10 & Algunas
poliat�micas
\\
{\sf \small Quiescent cores} & $10^4$ & 10 & Muchas poliat�micas
\\
{\sf \small Low mass young stellar objects} & & &
\\
{\sf \small Starless cores} & $10^4 - 10^6$ & 10 & Muchas
poliat�micas
\\
{\sf \small Collapsing cores} & $10^4 - 10^6$ & 10 & Agotamiento
por gr�nulos
\\
{\sf \small Protoplanetary disks} & $\leq 10^10$ & 30 &
Agotamiento por gr�nulos
\\
{\sf \small High mass young stellar objects} & & &
\\
{\sf \small Hot molecular cores} & $10^6$ & $100 - 300$ &
Mol�culas saturadas
\\
{\sf \small Shocked regions} & $10^4$ & $\leq 4000$ & Sputtering
\\
{\sf \small Photon-dominated regions} & $10^4$ & $100 - 1000$ &
Hidrocarburos Poliarom�ticos (PAH's)
\\\hline
\end{tabular}%\caption{Nubes Difusas \& Densas}
%\end{table}
\end{center}





\section{Another Section if wanted}
