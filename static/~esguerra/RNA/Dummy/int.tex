
\part{RNA conformational classes}

From the work of Schneider et al we see that they extract some conformational classes which we desire to analyze with 3DNA and analyze and compare.

\section{Article interpretation}
In Schneider \cite{schneider} we see 



\section{Analysis}
First describe what you did to the initial pdb files, then start thinking about what Wilma told you of the interpretation that can be done out of the parameters from 3DNA.
Describe what you have done to the data with the cluster analysis, why and possibly the method that the software is following to get the tree.


Mix the two articles:

\subsection{3DNA}

\begin{itemize}
\item [-]3DNA Applied to single stranded forms.
\item [-] N-M pair --> Shear Buckle reverse signs
\item [-] M+N pair opposite sign from N+M all 6 parameters
\item [-] Shear, Stretch, Opening ==> Hydrogen Bonding 
\item [-] Buckle, Propeller, Stagger ==> Secondary parameters, ``imperfections'', non-planarity


\end{itemize}



\subsection{RNA CONF CLASS}






\section{Results}





%\begin{figure}[htbp]
%\centering
%\includegraphics[scale=0.45]{pairing.png}
%\caption{Pairing Types}
%\end{figure}










